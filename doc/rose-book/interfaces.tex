\chapter{Interfaces}

\section{Standard Interfaces}

\subsection{Rose}

The {\em Rose} interface provides operations common to all other interfaces.  It is implicitly inherited by every interface.

\subsubsection{Operations}

\begin{itemize}
    \item \textbf{Destroy} Destroys the object associated with the capability.
    
\end{itemize}

\subsection{Rose.Capability}

Operations on capabilities.

\subsubsection{Inherits}

Rose

\subsubsection{Operations}

\begin{itemize}
    \item \textbf{Create Entry} Manufactures a capability which can call the given endpoint
    
    \begin{verbatim}
     function Create_Entry 
       (Endpoint : Rose.Objects.Endpoint_Id) 
       return Capability;    
    \end{verbatim}
    
    \item \textbf{Create Endpoint} Creates a new endpoint and returns its identity.
    
    \begin{verbatim}
     function Create_Endpoint
       return Endpoint_Id;    
    \end{verbatim}
    
    \item \textbf{Receive Any} Respond to an invocation of any known capability.
    
    \begin{verbatim}
        procedure Receive_Any
          (Called_Endpoint : out Endpoint_Id;
           Reply_Cap       : Capability;
           Parameters      : Invocation_Record);
           
    \end{verbatim}

    \item \textbf{Create Set} Manufactures a capability which represents a set of capabilities.
    
    \begin{verbatim}
     function Create_Set
       return interface Rose.Capability_Set;    
    \end{verbatim}
    
\end{itemize}

\subsection{Rose.Capability\_Set}

Operations on capability sets.

\subsubsection{Inherits}

Rose

\subsubsection{Operations}

\begin{itemize}
    \item \textbf{Insert} Adds a capability to the set
    
    \begin{verbatim}
     function Insert
       (Cap : Capability);    
    \end{verbatim}
    
    \item \textbf{Delete} Delete a capability from the set.
    
    \begin{verbatim}
        procedure Delete
          (Cap : Capability);
    \end{verbatim}

\end{itemize}

\subsection{Rose.Streams}

Operations on streams.

\subsubsection{Inherits}

Rose

\subsection{Rose.Streams.Read}

Operations for reading from streams

\subsubsection{Inherits}

Rose.Streams

\subsubsection{Operations}

\begin{itemize}
    \item \textbf{Read} Reads a buffer from the stream
    
    \begin{verbatim}
     procedure Read
       (Buffer : out Storage_Array;
        Length : in out Storage_Count)
    \end{verbatim}
\end{itemize}

\subsection{Rose.Streams.Write}

Operations for writing to streams

\subsubsection{Inherits}

Rose.Streams

\subsubsection{Operations}

\begin{itemize}
    \item \textbf{Write} Writes a buffer to the stream
    
    \begin{verbatim}
     procedure Write
       (Buffer : Storage_Array;
        Length : Storage_Count)
    \end{verbatim}
\end{itemize}

\subsection{Rose.Launch}

Launching processes.

\subsubsection{Operations}

\begin{itemize}
    \item \textbf{Launch} Launches the process named by this capability.  Sent caps are copied to the process.  Returns a capability which implements the Processes interface for this process.

    \begin{verbatim}
     function Launch
       (Caps : Array_Of_Capabilities)
       return interface Rose.Process;    
    \end{verbatim}
    
\end{itemize}

\subsection{Rose.Process}

Interface to processes.

\subsubsection{Operations}

\begin{itemize}
    \item \textbf{Launch} Launches the process named by this capability.  Sent caps are copied to the process.

    \begin{verbatim}
     function Launch
       (Caps : Array_Of_Capabilities)
       return Capability;    
    \end{verbatim}
    
\end{itemize}

\subsection{Rose.System.Create}

Interface for creating capabilities.

\subsubsection{Operations}

\begin{itemize}
    \item \textbf{Create Capability} Manufactures a capability based on the representation given
    in the sent words.
    
    \begin{verbatim}
     function Create_Capability
       (Header : Word_32;
        Payload_1, Payload_2, Payload_3 : Word_32)
       return Capability;    
    \end{verbatim}
    
\end{itemize}

\subsubsection{Derived Operations}

\begin{itemize}
    \item \textbf{Create Interface Capability} Manufactures an interface capability for the given endpoint and object.
    
    \begin{verbatim}
     function Create_Interface_Capability
       (Endpoint : Endpoint_Id;
        Object   : Object_Id)
       return Capability
     is (Create_Capability 
          (16#0000_0005#, Endpoint, 
           Low_Word (Object), High_Word (Object0)));
     
    \end{verbatim}
    
\end{itemize}

\subsection{Rose.System.Cap\_Copy}

Interface for copying capabilities from other objects.

\subsubsection{Operations}

\begin{itemize}
    \item \textbf{Copy Capability} Manufactures a capability which is a copy of an existing capability in the given object.
    
    \begin{verbatim}
     function Copy_Capability
       (Header : Word_32;
        Payload_1, Payload_2, Payload_3 : Word_32)
       return Capability;    
    \end{verbatim}
    
\end{itemize}

\subsection{Rose.System.Boot}

Boot module interface

\subsubsection{Operations}

\begin{itemize}
    \item \textbf{Launch} Starts a boot module.

    \begin{verbatim}
     function Launch
       (Module_Index : Positive;
        Caps         : Capability_Array)
       return Object_Id;
    \end{verbatim}
    
\end{itemize}

\subsection{Rose.System.Device\_Memory}

Interface for setting up memory-mapped devices.

\subsubsection{Operations}

\begin{itemize}
    \item \textbf{Reserve} Reserves an area of memory for devices.  Pages within this area can be acquired by device drivers using a page object cap.

    \begin{verbatim}
     procedure Reserve
       (Base  : Physical_Address;
        Bound : Physical_Address);
    \end{verbatim}
    
\end{itemize}

\begin{table}[ht]
\begin{tabular}{l r l l}
\hline\hline
Name & Identity & Implemented By & Notes \\ [0.5ex]
\hline
Meta            & 1 & All & A process usually has this as capability 1 \\
Cap Copy        & 2 & Kernel & Used during installation to subvert security \\
Kernel          & 3 & Kernel & \\
Procmem         & 4 & {\tt mm} & Paged process memory management \\
Registry        & 5 & registry & Registry of interface identifiers\\
Signal          & 6 & any & Send and receive signals \\
Authentication  & 7 & passwd & \\
Account         & 8 & passwd & \\
Make Executable & 9 & exec & \\
Executable      & 10 & ELF binaries & \\
Process         & 11 & running processes & \\
Login           & 12 & login & \\
User            & 13 & user & \\
\hline %inserts single line
\end{tabular}
\caption{Standard System Interfaces} % title of Table
\label{table:system_interfaces} % is used to refer this table in the text
\end{table}

\subsubsection{Meta Interface}

The {\em meta} interface is used for managing capabilities.  Sending to this interface creates new capabilities, or sends existing capabilities to other objects.  It can also be used to get a capability for an object's default interface.

\begin{figure}
\begin{bytefield}{32}
\bitheader[endianness=big]{0,7,8,11,12,15,16,21,22,25,26,31} \\
{\tiny
\bitbox{6}{Endpoint (1)}
\bitbox{4}{Alloc (0)}
\bitbox{6}{Caps (0)}
\bitbox{4}{SW (1)}
\bitbox{4}{RW (2)}
\bitboxes{1}{-SR-B{-}{-}{-}}} \\
\wordbox{1}{Capability (1)} \\
\wordbox{1}{Interface Id} \\
\bitbox{8}{Major Version}
\bitbox{8}{Minor Version}
\bitbox{8}{Release}
\bitbox{8}{} \\
\end{bytefield}
\caption{Interface capability request}
\end{figure}

\begin{table}[ht]
\begin{tabular}{l l l l}
\hline\hline
Endpoint & Name & Operations & Description \\ [0.5ex]
\hline
1 & Create Interface Cap & Send & Creates and returns a \\
  &                      &      & new interface capability \\
  &                      &      & for the sending object \\
2 & Create Endpoint Cap & Send & Creates and returns a \\
  &                     &       & new endpoint capability \\
  &                    &        & for the sending object \\
3 & Create Page Object Cap & Send & $data_0$ contains page address \\
 & & & $data_1$ contains the readable \\
 & & & and writable flags \\
5 & Create Singleton Receive & Send & \\
6 & Create Cap Set & Send & \\
16 & Receive on provided caps & Receive & Maximum receive caps is 6 \\
17 & Receive on any known cap & Receive & \\
31 & Exit & Send & End the sending process. \\
\hline
\end{tabular}
\caption{The meta interface}
\label{table:meta_interface}
\end{table}


\subsubsection{The process memory manager interface}

A server implementing the {\tt procmem} interface can be notified when a process starts, when it ends, and when it requests access to virtual addresses which it does not currently have.

It is expected that the server will have access to a page mapper interface, which will allow it to map and unmap pages as required.

\subsection{Authentication Interfaces}

\begin{table}[ht]
\begin{tabular}{l r l l}
\hline\hline
Name & Identity & Implemented By & Notes \\ [0.5ex]
\hline
Login           & 12 & login & \\
User            & 13 & user & \\
\hline %inserts single line
\end{tabular}
\caption{Standard Authentication Interfaces} % title of Table
\label{table:auth_interfaces}
\end{table}

\subsection{File Interfaces}

\begin{table}[ht]
\begin{tabular}{l r l l}
\hline\hline
Name & Identity & Implemented By & Notes \\ [0.5ex]
\hline

File System           & 100 & rfs, xmlfs, isofs & \\
Directory             & 101 & rfs, xmlfs, isofs & \\
File                  & 102 &  & \\
\hline %inserts single line
\end{tabular}
\caption{Standard File Interfaces} % title of Table
\label{table:file_interfaces}
\end{table}


\section{Interface Definition Language}

