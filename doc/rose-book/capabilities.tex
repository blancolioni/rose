\chapter{Capabilities}

\section{Introduction}

A capability represents permission to perform one or more operations.  The possession of a capability for an operation $O$ is both necessary and sufficient for performing $O$.

For the process, a capability is a 32-bit word, with no particular structure.  The kernel uses the process capability to index a private table of capability structures.  Each structure defines the capability type, and the object to which it applies.

Capability structures are stored in pages.  The first 32 capabilities used by a process are stored directly in the kernel process structure; the rest are stored in private pages mapped to the process.

\section{Capability Types}

\begin{table}[ht]
\begin{tabular}{l l}
\hline\hline
Type & Description \\
\hline
Null            &  The null capability; produces an error if used \\
Page Object     &  Controls access to physical memory \\
Schedule        &  \\
Process         &  \\
Interface       &  \\
Endpoint        &  \\
Set             &  \\
Kernel          &  \\
\hline
\end{tabular}
\caption{Capability Types}
\label{table:captypes}
\end{table}

\subsection{Generic Capability Layout}

A capability structure is 16 bytes long.  The particular layout depends on the capability type.  Common fields are as follows.

\begin{bytefield}{32}
\bitheader[endianness=big]{0,3,4,6,7,8,11,12,31} \\
\bitbox{20}{Identifier}
\bitbox{4}{Use$_{(4)}$}
\bitbox{1}{I}
\bitbox{3}{Flags}
\bitbox{4}{Type} \\
\wordbox{1}{Endpoint} \\
\wordbox{2}{Payload}
\end{bytefield}

The {\tt Identifier} is an identifier which must be matched when using a capability.  It is ignored unless the {\tt I} flag is set.

{\tt Use} is the remaining use count.  If this is zero, the capability can be used any number of times.  Otherwise, each use of the capability decrements {\tt Use} by 1, and when it reaches zero, the capability is overwritten with the null capability.

{\tt Flags} are capability-specific flags.

The {\tt Endpoint}, if non zero, is an index into the kernel-private endpoint array.

The {\tt Payload} is a type-specific payload.

\subsection{Endpoint Capabilities}

An endpoint capability is used to invoke an endpoint on the remote object.  If the process is currently listening on the capability's endpoint, the receive invocation will return with the corresponding invocation record.

\subsubsection{Endpoint Capability Layout}

\begin{bytefield}{32}
\bitheader[endianness=big]{0,3,4,7,8,15,16,31} \\
\bitbox{20}{Identifier}
\bitbox{4}{Use$_{(4)}$}
\bitbox{1}{I}
\bitbox{3}{0}
\bitboxes*{1}{0101} \\
\wordbox{1}{Endpoint} \\
\wordbox{2}{Object Id}
\end{bytefield}

\subsection{Page Object Capabilities}

A page object gives access to a shared buffer.  Three endpoints are implemented: {\tt copy-to-buffer}, {\tt copy-from-buffer}, and {\tt map-device-memory}.

\subsubsection{Page Object Capability Layout}

\begin{bytefield}{32}
\bitheader[endianness=big]{0,3,4,7,8,15,16,31} \\
\bitbox{20}{Identifier}
\bitbox{4}{Use$_{(4)}$}
\bitbox{1}{I}
\bitbox{1}{0}
\bitbox{1}{R}
\bitbox{1}{W}
\bitboxes*{1}{0100} \\
\wordbox{1}{Endpoint} \\
\wordbox{2}{Page Object Id}
\end{bytefield}

\subsection{Process Capabilities}

A process capability is used to communicate with a running process.  Two actions are possible, depending on the value of the {\tt Op} field.  If Op is zero, the capability returns the process's default interface.  If Op in one, the capability sends the capabilities in the invocation to the process.

Process capabilities are normally given to the parent process by the kernel when the child is launched.  The parent can then query the child, or send it capabilities.  For example, when the Petal shell executes a program, it uses the process capability to send the default input, output, and error streams (amongst other interfaces).

\subsubsection{Process Capability Layout}

\begin{bytefield}{32}
\bitheader[endianness=big]{0,3,4,7,8,15,16,31} \\
\bitbox{20}{Identifier}
\bitbox{4}{Use$_{(4)}$}
\bitbox{2}{0}
\bitbox{2}{Op}
\bitboxes*{1}{0100} \\
\wordbox{1}{0} \\
\wordbox{2}{Process Object Id}
\end{bytefield}

\subsection{Boot Capabilities}

These capabilities are only used during system initialisation.  At boot, the kernel launches the process {\em init}, which is pre-configured with a script which starts the rest of the system.  Boot capabilities are used to accomplish this.

\subsubsection{Boot Capability Endpoints}

\begin{table}[ht]
\begin{tabular}{l l l}
\hline\hline
Operation & Arguments \\ [0.5ex]
\hline
Launch Boot Module & Module index, caps \\
\hline
\end{tabular}
\caption{Boot capability endpoints} % title of Table
\label{table:boot-capability-endpoints}
\end{table}


\subsubsection{Boot Capability Layout}

\begin{bytefield}{32}
\bitheader[endianness=big]{0,3,4,7,8,15,16,31} \\
\bitbox{20}{Identifier}
\bitbox{4}{Use$_{(4)}$}
\bitbox{4}{0}
\bitboxes*{1}{1000} \\
\wordbox{1}{0} \\
\wordbox{1}{0} \\
\wordbox{1}{0} \\
\end{bytefield}

\subsection{Low Level Capabilities}

These are capabilities which allow device drivers to talk to hardware.  The are necessarily architecture-dependent.  Some examples are shown below.

\subsubsection{Port IO Layout (i686)}

This capability allows a process to execute the equivalent of {\tt out addr, reg} and {\tt in addr } instructions.

\begin{bytefield}{32}
\bitheader[endianness=big]{0,3,4,7,8,15,16,31} \\
\bitbox{20}{Identifier}
\bitbox{4}{Use$_{(4)}$}
\bitbox{2}{0}
\bitbox{2}{Sz}
\bitboxes*{1}{1110} \\
\wordbox{1}{Endpoint} \\
\bitbox{32}{First Port} \\
\bitbox{32}{Last Port} 
\end{bytefield}

The Sz field specifies 1-, 2-, 4- or 8- byte data for port in/out instructions (i.e. number of bytes per instruction is $2^{Sz}$).

The possible endpoints are as follows.

\begin{tabular}{l l l}
\hline\hline
Endpoint & Op & Description \\ [0.5ex]
\hline
1 &  port out & Each invocation word is sent to First Port \\
2 &  port in & Each receivable word is set by an in on First Port \\
3 &  port out range & invocation data specifies \\
  &                 & (port offset, value) pair for port-out \\
4 &  port in range & invocation data specifies port offsets, result written \\
   &                &to received words \\
\hline
\end{tabular}


\subsubsection{Invocation word format}

The sent words layout for the port-out-range endpoint varies depending on the size of the data.  For 8-bit data, each sent word consists of the value in byte 0, and the offset in byte 1.  16-bit data is encoded into bytes 0 and 1, with the offset in byte 2.  32-bit data is encoded in up to three groups of five sent words.  The first word of each group has four port offsets, while the second, third, fourth and fifth words contain the corresponding data for each port.

For the port-in-range endpoint, each sent word contains a port offset, and when returning the received words contain the result of the {\tt in} instruction.  If the received word count is larger than the sent word count, the extra offsets will all be zero.  Extra sent words are ignored.

In all cases, an offset of 255 indicates that no data should be sent.  Any offset that is not within the port range is also ignored.

\section{Invocation}

A capability is invoked by making a system call using an {\em invocation record}.

\subsection{Invocation Record}

\begin{bytefield}{32}
\bitheader[endianness=big]{0,1,2,3,4,5,8,11,12,15,16,19,20,23,24,31} \\
\bitbox{8}{reserved}
\bitbox{4}{Use}
\bitbox{4}{CC}
\bitbox{4}{RWC}
\bitbox{4}{SWC}
\bitboxes{1}{WPESRYBC} \\
\wordbox{1}{Capability} \\
\wordbox{1}{Reply Capability} \\
\wordbox{1}{Shared buffer length} \\
\wordbox{1}{Shared buffer address} \\
\wordbox{5}{Up to 15 capabilities} \\
\wordbox{5}{Up to 15 argument words}
\end{bytefield}

The {\tt RWC} field is the number of words which the sender is prepared to receive in response to the invocation.  A reply to this message cannot contain more words than specified here, although it can contain less.

The {\tt SWC} field is the number of sent words.

The {\tt CC} field is the number of sent words.

The {\tt Capability} field contains the capability being invoked.  The {\tt Reply Capability} field is set by the kernel, and gives the target of an invocation a capability on which to send a reply.

If any capabilities are sent, the {\tt Use} field limits the number of times they can be used.  An allocation count of zero means there is no limit.

\begin{table}[ht]
\begin{tabular}{l l l}
\hline\hline
Id & Name & Description \\
\hline
W & Write & If the P flag is set, allow the receiver to write to the buffer \\
P & Pages & Send one or more shared buffers \\
E & Error & An error occurred (code is in first data word) \\
S & Send & Unprompted send on a capability \\
R & Receive & Willing to receive messages \\
Y & Reply & This is a reply to an earlier send \\
B & Block & The sender will block until the next message \\
C & Cap Construct & Create a reply cap for the message response \\
SWC & Sent Word Count & SWC contains the number of sent words \\
RWC & Receive Word Count & RWC contains the maximum received words \\
CC & Capability Count & Number of sent words which are capabilities \\
\end{tabular}
\caption{Invocation Flags}
\label{table:invocation_flags}
\end{table}

