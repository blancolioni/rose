\chapter{Memory}

Physical memory is managed by the memory manager.  Page objects are managed by the page object manager.  A process belongs to a particular space bank, which represents a range of page objects.  Space banks are hierarchical.  A page object is a direct member of at most one space bank.  At the top of the space bank hierarchy is the root space bank, which encompasses the entire page object id range.

Each process executes in a flat memory space starting at zero.  These are the virtual pages.  Virtual pages are mapped or unmapped.  A mapped virtual page references a corresponding physical page.  A physical page may be mapped by any number of virtual pages.  This mapping is not persisted.

\section{Example}

The command {\tt ls} lists the contents of a directory.  A user shell normally has a launch capability for {\tt ls}.  When this capability is invoked, a new process is created.  A snapshot of part of the memory state might look something like table \ref{table:ls-launch}.

\begin{table}[ht]
\begin{tabular}{l l l l}
\hline\hline
Page Object & Virtual Address & Physical Address & Description \\
\hline
$ls + 0$ & {\tt 0000 0000} & {\tt 4020 C000} & 1st page of ls text segment \\
$ls + 1000$ & {\tt 0000 1000} & {\tt ABCD 2000} & 2nd page of ls text segment \\
$ls + 2000$ & {\tt 0000 2000} &  & 3rd page (not paged in yet) \\
$base + 3000$ & {\tt BFFF F000} & {\tt D020 1000} & stack page (in user space map) \\
\hline
\end{tabular}
\caption{Example memory snapshot after launching {\tt ls}}
\label{table:ls-launch}
\end{table}

Every physical page is either free, or mapped to a page object.  This mapping is not persisted.  The memory manager keeps track of free memory pages and memory page $\Leftrightarrow$ page object mappings.

\section{Memory Interface}

A memory capability can be used to map a page object id.  This makes the page object available to a running process, although it does not necessarily exist in physical memory yet.  When a process is launched, four page object ranges are mapped: the code segment (read-only, executable), the text segment (read-only), the data segment (read/write) and the stack segment (read/write).

If the memory server already knows about a read-only page object id, the map is not updated.

A stack segment page object id is always new, and is therefore always recorded in the memory manager.

A data segment page object may be initialised or uninitialised.

If an initialised read/write page object id is mapped, and the memory manager already has the page object mapped read-only, the page object is mapped to the read-only version.  If it is later written to, a copy will be made and the page object will be re-mapped.

An uninitialised read/write page object is recorded (same as a stack segment page).
