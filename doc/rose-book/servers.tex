\chapter{Servers}

\section{Boot Servers}

When the kernel is ready to schedule its first process, it chooses the first boot module.  This is the {\em init} program.  Its job is to load the rest of the boot modules, and find the most recent consistent checkpoint from which to restore.

Init requires a particular set of launch caps (see Table \ref{table:init-launch-caps}).  Each of these capabilities is invoked in order.  Init runs with a low priority, causing it to block while each boot server initialises.

Init is configured via a Petal script (see Figure \ref{fig:init.petal} for a small example).  This script is compiled to an Ada package spec, which is with'd by the {\tt init} source.


\begin{table}[ht]
\begin{tabular}{l l l}
\hline\hline
Cap & Name & Description \\
\hline
1 & meta & Standard meta cap \\
2 & launch & Cap for launching boot modules \\
2 & copy & Cap for copying process caps \\
3 & construct-port & Cap for constructing port caps \\
\hline
\end{tabular}
\caption{Launch caps for {\em init} process}
\label{table:init-launch-caps}
\end{table}

\begin{table}[ht]
\begin{tabular}{l l l l}
\hline\hline
Pid & Name & Interface & Description \\
\hline
2 & init & none & Installs system image and launches boot servers \\
3 & console & stream-writer & writes to boot terminal \\
4 & hd0 & storage & boot hard drive \\
7 & mem & mem & memory manager \\
8 & checkpoint & checkpoint & checkpointing process \\
9 & proc & proc & process manager \\
\hline
\end{tabular}
\caption{Boot Servers}
\label{table:boot_servers}
\end{table}

\subsection{Memory Manager}


\subsection{Process Manager}
