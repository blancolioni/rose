\chapter{Boot Sequence}

\section{Overview}

The boot process initialises the kernel and starts the first process, {\tt init}.

\section{Initial Processes}

\subsection{{\tt init}}

The kernel gives the init process a single {\tt create} capability, which is used to create launch and copy capabilities for the boot modules.

\subsection{console}

The console program implements a stream writer interface, and sends anything it receives to its device.  For rose-x86 it writes directly to the console buffer, which is mapped by the kernel at boot time.  Normally this is the first server launched by {\tt init}.

\subsection{cap set}

The cap set process allows other processes to create a set of capabilities, which can be referenced using a single capability.  Once this process is running, the remaining boot processes are not limited to four capabilities each.

\subsection{timer}

The kernel provides a single timeout, so there is a timer server which multiplexes multiple timers onto it.

\subsection{mem}

The standard memory manager is launched with a memory capability, which can be queried for the available physical memory layout, and invoked to map and unmap virtual addresses to physical ones.  It implements the {\tt memory manager} interface, which can add and remove processes, and handle page faults.

\subsection{pci}

Scans the PCI bus.

\subsection{ata}

Drive for ATA/ATAPI storage.

\section{First Boot}

\section{Restore Boot}

\section{Scripts}

The boot process is controlled by a compiled Petal script.  A sample script is shown in figure {\ref{fig:init.petal}}.

\begin{figure}
    \centering
\begin{verbatim}
    procedure Init
      (Create    : interface Meta)
    is
       Console : constant interface Process :=
         Launch.Launch_Boot_Module (2);
       Writer  : constant interface Stream_Writer :=
         Cap_Copy.Copy_Cap (Console, 1);
    begin
       Output_Buffer := Start_Init;
       Writer.Write (Output_Buffer, Start_Init'Length);
    end Init;
\end{verbatim}
\caption{Sample Petal script for init}
    \label{fig:init.petal}
\end{figure}

