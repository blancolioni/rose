\chapter{Processes}

A process is named by a launch capability.  Invoking the capability executes the process.

Object ids in the range $0$ to $2^{24} - 1$ represent processes.

Process id 1 is used to refer to the Kernel.

\section{Launch Capability}

\section{Initial Capabilities}

The generated interface libraries rely on caps 1 and 2 being the ones from table \ref{table:standard-caps}.

The set referenced by cap 2 is defined by Table \ref{table:standard-cap-env}.
The environment accessed by the Ada.Environment\_Variables package contains an entry for each {\em Id} in the table, with the index of the corresponding capability in the cap set.
The standard libraries rely on cap 7 of the cap set being a capability for the {\tt Cap\_Set} interface, and cap 6 being the delete capability method.

\begin{table}[ht]
\begin{tabular}{l l l l l}
\hline\hline
\# & Cap & Type & Description \\
\hline
1 & Self & Interface & Process interface to current process \\
2 & Argument\_Caps & Interface & Cap set containing supplied caps \\
\hline
\end{tabular}
\caption{Standard process capabilities}
\label{table:standard-caps}
\end{table}

\begin{table}[ht]
\begin{tabular}{l l l l}
\hline\hline
\# & Id & Interface & Description \\
\hline
1 & Standard\_Input & Stream\_Reader & Standard input stream \\
2 & Standard\_Output & Stream\_Writer & Standard output stream \\
3 & Standard\_Error & Stream\_Writer & Standard error stream \\
4 & Current\_Directory & Directory & Start directory \\
5 & Clock & Clock & System clock \\
6 & Delete\_Cap & Method & Delete capability \\
7 & Rescind\_Cap & Method & Rescind capability \\
8 & Create Endpoint & Method & Endpoint constructor \\
9 & Create Cap\_Set & Method & Cap set constructor \\
\hline
\end{tabular}
\caption{Standard capability environment}
\label{table:standard-cap-env}
\end{table}

\section{Example}

Consider the following excerpt from a shell session:

{\tt user\$ cat readme.txt }

A launch capability for the {\tt cat} program is found in the shell's environment.  A new environment is created, setting environment capabilities for input/output streams.  The argument is interpreted as the name of a file, a {\tt stream-reader} capability to this file is added to the environment.

\section{Environment}

The environment is supplied as a table of entries starting at a known address.

